\documentstyle[12pt]{article}
\setlength{\oddsidemargin}{12pt}
\setlength{\textwidth}{6.5in}
\setlength{\textheight}{9in}
\pagestyle{empty}
\setlength{\parskip}{7pt plus 2pt minus 2pt}

\begin{document}

\begin{center}
{{\large CS 230 : Discrete Computational Structures}}\\

\vspace*{1cm}

{\bf Fall Semester, 2024}\\

{\sc Homework Assignment \#1}\\
{\bf Due Date:}  Friday, September 6
\end{center}

\noindent {\bf Suggested Reading:} Rosen Sections 1.1 - 1.3; LLM Sections 1.1,  3.1 - 3.4

These are the problems that you need to hand in for grading. Always
explain your answers and show your reasoning.

\begin{enumerate}

\item {\bf [6 Pts]} Translate the following English sentences into logic. First, define your basic propositions and use logical operations to connect them.\\
        (a) You will pass the class only if you attend class regularly and do your homework.\\
        (b) Being a good speaker is sufficient for being elected as president and for being a successful teacher.\\
				(c) Being a good programmer and having good communication skills are necessary to get a good job. 
				
\item {\bf [6 Pts]} State the converse, inverse and contrapositive of  the statement ``if it is Sunday, you go to a movie.''				
				
\item {\bf [6 Pts]} Determine whether $((p \rightarrow q) \wedge (q \rightarrow r)) \leftrightarrow (p \rightarrow r)$ is a tautology using truth tables.

\item {\bf [6 Pts]} Prove that $(p \rightarrow (q \rightarrow r))$ and $(q \rightarrow (p \rightarrow r))$ are logically equivalent by deduction using a series of logical equivalences studied in class (truth tables will not be allowed).

\item {\bf [10 Pts]} Use logical reasoning to solve the following puzzle:
\\
{\it A detective has interviewed four witnesses to a crime. From the stories of the witnesses the detective has concluded that if the butler is telling the truth then so is the cook. The cook and the gardener are either both telling the truth or they are both lying. Either the gardener or the handyman is telling the truth, but not both. If the handyman is lying then so is the cook. Determine which witnesses are lying and which are telling the truth. Define four basic propositions and use these to describe the four statements above. Then use deductive reasoning, not truth tables, to derive the truth of each of those basic propositions.
}\\
Hint: Start by assuming that the cook is telling the truth. This leads to a contradiction, so you conclude that the cook is lying. Now, you can go on to conclude the facts about the other three witnesses.

\item {\bf [6 Pts]} Prove that $\{\leftrightarrow, \vee, \rm FALSE\}$ is functionally complete, i.e.,any propositional formula is equivalent to one whose only connectives are $\leftrightarrow$ and $\vee$, along with the constant FALSE. Prove using a series of logical equivalences. You may assume any logical equivalences we studied in class and the fact that any formula is equivalent to some formula in DNF. {\it Note: If you make the statement that a particular set of operators is functionally complete, and use this in your proof, then you need to justify your statement.}

\item {\bf [10 Pts]} Consider the proposition $(p \vee q) \rightarrow (\neg q \wedge r)$. 
\begin{enumerate}
\item Construct a truth table for the proposition above.
\item We showed in class that any compound proposition is logically equivalent to some proposition in DNF. Use the procedure to construct a DNF proposition that is equivalent to the proposition above, starting with the truth table (show your work).
\item A proposition is said to be in CNF (conjunctive normal form) if it is a conjunction (and) of one or more clauses, where each clause is a disjunction (or) of basic propositions or their negations. For example, $(p \vee \neg q) \wedge (\neg p \vee \neg q \vee r) \wedge (q \vee r)$ is in CNF. Any compound proposition is also logically equivalent to some proposition in CNF. Come up with a procedure to construct a CNF proposition that is equivalent to the proposition above, starting with the truth table (show your work).
\end{enumerate}

\end{enumerate}

For more practice, you are encouraged to work on the problems given at the end of Rosen, Sections 1.1 - 1.3.

\end{document}


